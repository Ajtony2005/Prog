\documentclass[10pt, conference,a4paper]{ITKproc}
\documentclass{article}
% \usepackage[normalem]{ulem}
% \useunder{\uline}{\ul}{}
\usepackage[utf8]{inputenc}
\usepackage{graphicx}
\usepackage{amsmath}
\usepackage{amssymb}
\usepackage{hyperref}
\usepackage[magyar]{babel}
\usepackage{circuitikz}

% correct bad hyphenation here

\hyphenation{pre-sence vi-su-alized si-mu-la-tions mo-le-cu-lar se-ve-ral cha-rac-te-ris-tic CoNSEnsX}


\begin{document}
%
% paper title
% can use linebreaks \\ within to get better formatting as desired
\title{Gps jegyzőkönyv}

\author{\IEEEauthorblockN{Szauter Ajtony, Zsumbera Olivér}
\IEEEauthorblockA{(Supervisor: Naszlady Márton Bese)\\
Pázmány Péter Katolikus Egyetem, Információs Technológiai és Boinikai Kar\\
 1083, Budapest Práter utca 50/A\\
2024.04.10 \\
\texttt{szauter.ajtony@hallgato.ppke.hu}}
\texttt{zsumbera.oliver@hallgato.ppke.hu}}
}

% make the title area
\maketitle

\begin{abstract}
Megismerkedés a GPS-el. Mérés a friss tiszta levegőn a BUdapesten lévő Corvin negyedben a Práter Utca 50.ben lévő Pázmány Péter Katolikus Egyetem Információ Technológia és Bionika kara épülete kürül.

\end{abstract}
\begin{IEEEkeywords}
Gps elmélete, története, müködése és mérése
\end{IEEEkeywords}
\IEEEpeerreviewmaketitle

\section{Gps}

A GPS (Global Positioning System) rövidítés a Globális Helymeghatározó Rendszert takarja, mely műholdak segítségével határozza meg a felhasználók földrajzi pozícióját. Manapság már elterjedt fogalom, de sokan nem tudják pontosan, hogyan is működik.

A GPS rendszert az Egyesült Államok fejlesztette ki, a műholdak pontos pályainformációkat sugároznak, amiket a vevők fogadnak és számítással meg tudják határozni a távolságukat a műholdaktól. A távolságmérés alapján, trigonometriai eljárással meg lehet határozni a vevő pontos pozícióját a Földön.


Jelenleg négy műholdas helymeghatározó rendszer is létezik (GNSS, Global Navigational Satellite System):

Az amerikai NAVSTAR GPS (NAVigational Satellites for Timing And Ranging)
Az orosz GLONASSZ rendszer
Az Európai Uniós Galileo
A kínai Compass (korábban Beidu)
A Galileo rendszer kiépítése még zajlik, a NAVSTAR GPS viszont teljes lefedettséggel rendelkezik.

A műholdas helymeghatározó rendszer működési elve
A GPS rendszer műholdak sokaságából áll, melyek pontosan ismert pályákon keringenek a Föld körül. A vevő a műholdak által sugárzott jelek alapján méri a távolságát a műholdaktól. A távolságméréshez szükséges a műhold óra pontos ismerete, ami atomórákon alapul.

Mivel a Föld forog, a műholdak pályája is folyamatosan változik, ezért a rendszer pontos időjelzéseket is ad, amikkel szinkronizálni lehet a vevő óráját.

A pontos helymeghatározáshoz legalább négy műhold jele szükséges. A vevő a mért távolságok alapján kiszámítja a pozícióját a Földön. A számítás trigonometriai eljáráson alapul, azaz a vevő és a műholdak közötti távolságokból háromszögeket szerkesztve határozza meg a pozícióját.

A GPS rendszer nagyon pontos, a mérési hiba általában néhány méter nagyságrendű. A pontosság függ a műholdak geometriájától, a vevő típusától és a környezeti feltételektől.

A GPS rendszer alrendszerei
A GPS rendszer három fő alrendszerből áll:

Műholdak alrendszere: Jelenleg 24 műholdból áll, melyek 6 pályán keringenek a Föld körül. A műholdak jeleket sugároznak, amik a Földön lévő vevők számára elérhetőek.
Földi állomások alrendszere: Öt ismert koordinátájú állomásból áll, melyek a műholdak pályáját követik és adatokat gyűjtenek róluk. Ezek az adatok szükségesek a műholdak pontos pozíciójának és órajelének meghatározásához.
Felhasználók alrendszere: Tetszőleges számú GPS vevőberendezést foglal magában. A vevők fogadják a műholdak jeleit, és kiszámítják a pozíciójukat.
A GPS mérési módszerek
Két fő GPS mérési módszer létezik:

Időméréses (kódméréses): A vevő a műhold által sugárzott kód alapján méri a távolságot a műholdig. Ez a módszer kevésbé pontos, mint a fázisméréses módszer, de kevésbé érzékeny a környezeti zavarokra.
Fázisméréses: A vevő a vivőhullám fázisának mérésével határozza meg a távolságot a műholdig. Ez a módszer pontosabb, mint az időméréses módszer, de érzékenyebb a környezeti zavarokra.
A mérési módszertől függően a GPS rendszer különböző pontosságot nyújthat. A kódméréses módszerrel 1-5 méteres pontosság érhető el, míg a fázisméréses módszerrel mm pontosság.


A GPS rendszer alkalmazásai
A GPS rendszert számos területen alkalmazzák, ahol fontos a pontos helymeghatározás. Íme néhány példa:

Navigáció: A legismertebb felhasználási terület a navigáció. A GPS alapú autós navigációs rendszerek, túra GPS-ek és okostelefonok segítenek eljutni a kívánt célállomásra.
Földmérés: A GPS-t a földmérők is használják a pontok koordinátáinak meghatározásához.
** Térképészet:** A GPS segítségével pontosabb és részletesebb térképeket lehet készíteni.
Közlekedés:​ A GPS-t a szállítási útvonalak optimalizálására és a járművek valós idejű követésére használják.
​*Időzóna-beállítás:​ A GPS-vevők automatikusan beállíthatják az időzónát a felhasználó földrajzi helyzete alapján.
​*Mentés:​ A GPS- alapú segélyhívások segíthetik a bajba jutott személyek megtalálását.
​*Sport és fitness:​ A futók, kerékpározók és egyéb sportolók a GPS-t használhatják az edzéseik nyomon követésére.
A GPS technológia folyamatosan fejlődik, és várhatóan egyre több új alkalmazási terület fog megjelenni a jövőben.


\end{document}